\section*{Introduction}
\addcontentsline{toc}{chapter}{Introduction}

Au fil de mes études, je me suis passionné pour les mathématiques appliquées, la mécanique des fluides et j’avais hâte de découvrir le monde de la recherche. Lors d’une présentation des activités au \ac{CERFACS} par nos deux enseignants chercheurs, Arthur COLOMBIÉ et Nadir MESSAI, j’ai découvert ce laboratoire. Suite à un échange plus approfondi avec eux,  je n’ai pas hésité à candidater pour mon stage de M1 et j’ai eu la chance d’être retenu.
Lorsque Carlos MONTILLA, docteur au CERFACS, m’a proposé un sujet sur l’interpolation dans le cas de post-traitement de simulations numériques et son application en aéroacoustique, cela m’a séduit et motivé. C’est donc avec enthousiasme que j’ai débuté mon stage le 10 juin 2024 dans ce laboratoire, pour une durée de trois mois.
En préambule, j’aimerais préciser que le secteur de la recherche en calcul scientifique est réputé être exigeant, car à la base de l’innovation. C’est ce qui, à mes yeux, le rend si stimulant.

En quelques mots, Le CERFACS est un institut de recherche privé, financé par sept actionnaires. Il est spécialisé et reconnu pour son expertise en \ac{HPC}. C’est un acteur principal dans ce domaine, en collaboration avec des industriels et des institutions renommés et connus pour développer des solutions à la pointe de la technologie.

Mon maître de stage, Carlos MONTILLA, est responsable d’Antares \cite{antares} depuis un an. Antares est un code d’analyse de données privé sous forme de librairie python et C++, développé au CERFACS depuis 2012 et dont l’objectif est de réaliser du pré et post-traitement sur des simulations numériques utilisées par le laboratoire, ses actionnaires et autres partenaires. Il contient notamment un traitement d’interpolation et un traitement \ac{FWH}, principalement codé en Python par Carlos MONTILLA. La chaîne de calcul aéroacoustique (sous-section \ref{s243}) utilise le traitement d’interpolation avant de pouvoir utiliser le traitement FWH. L’objectif de l’interpolation d’Antares est de pouvoir interpoler les valeurs aux points d’un maillage ’cible’, issu d’une discrétisation de l’espace, en utilisant les valeurs aux points d’un maillage ’source’ ; par exemple, dans le cadre d’un raffinement de maillage entre deux itérations de calcul ou lors de la création d’une surface de contrôle dans un maillage 3D pour appliquer les équations de FWH à la propagation aéroacoustique.

Mes missions principales lors de ce stage ont été les suivantes :

\begin{itemize}
    \item réaliser un état de l'art sur les autres méthodes d'interpolation potentiellement implémentables dans Antares (avec les contraintes associées);
    \item identifier les meilleurs paramètres pour l'équation \ac{IDW} la seule qui était implémentée jusqu'alors dans Antares;
    \item implémenter la méthode linéaire, de la tester et d'optimiser le code.  % \ac{DSP} à placer
\end{itemize}

% A savoir, 3D, non structuré, temps de calcul, caractéristiques des équations à interpoler ...

Pour expliquer plus en détail le contenu du stage, je présenterai dans une première partie le laboratoire de recherche CERFACS. Dans une seconde partie j’exposerai le travail que j’ai réalisé sous forme de rapport. Cette seconde partie se décompose en quatre sous-parties : la présentation de la librairie Antares, les différentes méthodes d'interpolation, l'implémentation de la méthode linéaire dans Antares et finalement les tests de l'ancienne et de la nouvelle méthode d'interpolation.