\section*{Introduction}
\addcontentsline{toc}{chapter}{Introduction}

J'adore les mathématiques appliquées, la mécanique des fluides et je voulais découvrir le monde de la recherche. Lors d'une présentation des activités au
\ac{CERFACS} par nos deux enseignants chercheurs Arthur et Nadir, j'ai eu l'occasion de découvrir ce laboratoire et d'y candidater pour mon stage de M1. Carlos m'a trouvé un sujet sur l'interpolation dans le cas de post-processing de simulations numériques et son application en aéroacoustique. Le sujet m'a directement plu, j'ai ainsi pu commencer mon stage le 10 juin 2024 au CERFACS.

En quelques mots, le CERFACS est un institut de recherche privé, spécialisé dans le développement de code \ac{HPC}, financé par sept actionnaires.

Antares\cite{antares} est une librairie python privée qui a été développée au CERFACS en 2012 et a pour objectif de faire du pré et post-processing sur des simulations numériques utilisées par les actionnaires du CERFACS.

Elle contient notamment une fonction d'interpolation, codée en Python.

Mon maître de stage, Carlos, est responsable d'Antares depuis dix mois. 

Mes missions principales lors de ce stage ont été :

- De faire un état des lieux sur les autres méthodes d'interpolation qui seraient 
implémentables dans Antares (avec ses contraintes associées).
% A savoir, 3D, non structuré, temps de calcul, caractéristiques des équations à interpoler ...

- D'identifier les meilleurs paramètres pour l'équation \ac{IDW} % : Inverse Distance Weighting, pondération inverse à la distance en français}

- D'implémenter la méthode trilinéaire que j'appellerais aussi barycentrique.

Pour bien comprendre, ce que nous voulons interpoler, ce sont les valeurs aux points d'un maillage dit 'target' grâce aux valeurs aux points d'un maillage 'source'. Par exemple dans le cadre d'un raffinement de maillage entre 2 itérations de calcul ou dans le cas de la création d'une sphère dans un maillage 3D pour l'application des équations de \ac{FWH} dans le cadre de la propagation aéroacoustique.

Pour expliquer plus en détails ce stage au CERFACS, je présenterai d'abord ce laboratoire, puis je vous exposerai le travail que j'ai réalisé.

 % 
% 