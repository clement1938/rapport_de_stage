\section*{Bilan Technique}
\addcontentsline{toc}{chapter}{Bilan Technique}
%\begin{comment}
Ce bilan technique a pour objectif de synthétiser les écarts et leurs causes entre les missions du stage et ce qui a été réalisé.
Il permet aussi de proposer ce qui pourrait être fait par la prochaine personne travaillant sur le même sujet.

\begin{table}[ht]
\centering
\begin{tabular}{|p{6.5cm}|p{8.5cm}|}
\hline


\multicolumn{2}{|c|}{\textbf{Fiche de synthèse}   \hspace{7cm}   Clément THIBAULT - Aéro 4} \\ 
\hline
\textbf{Sujet de stage} & \textbf{Objectifs} \\ 
\hline


\begin{minipage}[t]{6.5cm}
Influence de la méthode d’interpolation sur la propagation acoustique FWH
\end{minipage} & 
\begin{minipage}[t]{8.5cm}
%\begin{itemize}
- Développer une méthode d’interpolation linéaire HPC dans Antares

- Évaluer l’influence de la méthode d’interpolation dans la qualité des résultats de propagation acoustique avec l’analogie FWH

- Améliorer les performances HPC de la méthode d’interpolation dans Antares
%\end{itemize}
\end{minipage} \\ 
\hline
\textbf{Client principal} & \textbf{Outils utilisés} \\ 
\hline
\begin{minipage}[t]{6.5cm}
%\begin{itemize}
- CERFACS

- Date de mise à jour : 15 octobre 2024
%\end{itemize}
\end{minipage} & 
\begin{minipage}[t]{8.5cm}
VSCode, Python, Kraken (supercalculateur du CERFACS), Antares, Paraview, Git
\end{minipage} \\ 
\hline
\multicolumn{2}{|l|}{\textbf{Études réalisées}} \\ 
\hline
\multicolumn{2}{|p{14cm}|}{
\begin{minipage}[t]{14cm}
\begin{itemize}
    \item Influence de paramètres de la méthode d'interpolation IDW
    \item Différences de DSP entre la méthode IDW et linéaire
    \item Optimisation de la rapidité du traitement
    \hspace{0.5cm}
\end{itemize}
\end{minipage}
} \\ 
\hline
\textbf{Résultats} & \textbf{Explications des écarts possibles} \\ 
\hline
\begin{minipage}[t]{6.5cm}
%\begin{itemize}
%\raggedright
%\justifying

- Meilleurs paramètres pour l'IDW se situent autour de N=10 et p=10

- Méthode linéaire généralement plus précise que IDW

- Rapidité du code d'interpolation augmentée, pour toutes les méthodes, d'un facteur 100 sur le cas test d'aéroacoustique

%\item Paramètres n \simeq 10 et p \simeq 10 optimaux pour la méthode IDW
%\end{itemize}
\end{minipage} & 
\begin{minipage}[t]{8.5cm}

- N=10 et p=10 donnent beaucoup d'information et d'importance aux points proches

- La méthode linéaire est d'ordre 1 contrairement à IDW qui n'a pas d'ordre au sens usuel du terme

- Pour la rapidité, une amélioration a consisté à ne pas recalculer des coefficients à chaque instant de la solution

\end{minipage} \\ 
\hline
\textbf{Difficultés rencontrées} & \textbf{Travaux à poursuivre} \\ 
\hline
\begin{minipage}[t]{6.5cm}
%\begin{itemize}
- Prise en main des outils

- Adapter la méthode linéaire au code déjà existant
%\end{itemize}
\end{minipage} & 
\begin{minipage}[t]{8.5cm}
%\begin{itemize}
- Optimiser la méthode linéaire pour les maillages prismatiques

- Traiter les maillages 'multi-zones' avec points partagés en linéaire

- Implémenter une méthode d'ordre supérieur

- Passer le code en parallèle pour améliorer la rapidité sur supercalculateur
%\end{itemize}
\end{minipage} \\ 
\hline
\end{tabular}
\end{table}  
%\end{comment}
\newpage
