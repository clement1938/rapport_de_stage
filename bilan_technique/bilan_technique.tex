\section*{Bilan Technique}
\addcontentsline{toc}{chapter}{Bilan Technique}
%\begin{comment}
\begin{table}[ht]
\centering
\begin{tabular}{|p{8cm}|p{8cm}|}
\hline
\multicolumn{2}{|c|}{\textbf{Fiche de synthèse}   \hspace{7cm}   Clément THIBAULT - Aéro 4} \\ 
\hline
\textbf{Sujet de stage} & \textbf{Objectifs} \\ 
\hline
\begin{minipage}[t]{8cm}
Influence de la méthode d’interpolation sur la propagation acoustique FWH\footnote{FWH:.. }
\begin{itemize}
    \item Développer une méthode d’interpolation trilinéaire HPC dans le code d’analyse de données Antares
    \item Évaluer l’influence de la méthode d’interpolation dans la qualité des résultats de propagation acoustique avec l’analogie Ffowcs Williams – Hawkings
    \item Améliorer les performances HPC de la méthode d’interpolation dans Antares
\end{itemize}
\end{minipage} & 
\begin{minipage}[t]{8cm}
\begin{itemize}
    \item Développer une méthode d’interpolation trilinéaire HPC dans le code d’analyse de données Antares
    \item Évaluer l’influence de la méthode d’interpolation dans la qualité des résultats de propagation acoustique avec l’analogie Ffowcs Williams – Hawkings
    \item Améliorer les performances HPC de la méthode d’interpolation dans Antares
\end{itemize}
\end{minipage} \\ 
\hline
\textbf{Client principal} & \textbf{Outils utilisés} \\ 
\hline
\begin{minipage}[t]{8cm}
\begin{itemize}
    \item CERFACS (AIRBUS, Safran)
    \item Date de mise à jour : octobre 2024
\end{itemize}
\end{minipage} & 
\begin{minipage}[t]{8cm}
VSCode, Kraken (supercalculateur du\\
CERFACS), Antares, Paraview
\end{minipage} \\ 
\hline
\multicolumn{2}{|l|}{\textbf{Études réalisées}} \\ 
\hline
\multicolumn{2}{|p{14cm}|}{
\begin{minipage}[t]{14cm}
\begin{itemize}
    \item Influence des paramètres 'n' et 'p' de la méthode d'interpolation 'IDW'
    \item ...
\end{itemize}
\end{minipage}
} \\ 
\hline
\textbf{Résultats} & \textbf{Explications des écarts possibles} \\ 
\hline
\begin{minipage}[t]{8cm}
\begin{itemize}
    \item Rapidité du code d'interpolation augementée (quelque soit la méthode), environ par 100 sur le cas test d'aéroacoustique
    \item Méthode trilinéaire pour tous types de maillages rencontrés au CERFACS implémentée
    \item Méthode trilinéaire généralement plus efficace que la méthode 'IDW' dans les cas d'aéroacoustique
    %\item Paramètres n \simeq 10 et p \simeq 10 optimaux pour la méthode IDW
\end{itemize}
\end{minipage} & 
\begin{minipage}[t]{8cm}
\begin{itemize}
    \item ...
    \begin{itemize}
        \item ...
    \end{itemize}
\end{itemize}
\end{minipage} \\ 
\hline
\textbf{Difficultés rencontrées} & \textbf{Travaux à poursuivre} \\ 
\hline
\begin{minipage}[t]{8cm}
\begin{itemize}
    \item Prise en main des outils relativement fatiguante au début
    \item Echec de l'implémentation du\\
    'multi-zones' pour l'interpolation\\
    trilinéaire
\end{itemize}
\end{minipage} & 
\begin{minipage}[t]{8cm}
\begin{itemize}
    \item Implémenter une méthode d'ordre supérieur dans le code d'Antares
\end{itemize}
\end{minipage} \\ 
\hline
\end{tabular}
\end{table}  
%\end{comment}
\newpage
