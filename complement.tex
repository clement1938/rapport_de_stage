\documentclass[11pt,a4paper]{report}
\usepackage{tikz}
\usepackage{amsmath}
\usetikzlibrary{patterns}

\begin{document}


Document en complément de la partie 2.3.3 (Optimisation du temps de calcul) du rapport de stage de Clément THIBAULT (2024) au CERFACS. Il s'agit de décrire une méthode pour savoir si une surface créé par deux points d'une cellule triangulaire et un point à interpoler est compris dans la cellule ou non.

\begin{tikzpicture}
    % Définition des points
    \coordinate (A) at (0, 0);
    \coordinate (B) at (4, 0);
    \coordinate (C) at (2, 3);
    \coordinate (P) at (2.5, 5); % Point P, placé en dehors de deux droites

    % Tracer le triangle ABC
    \draw[thick] (A) -- (B) -- (C) -- cycle;

    % Tracer les droites étendues des segments du triangle
    \draw[dashed] (A) -- ++(-2,-3); % Droite étendue
    \draw[dashed] (A) -- ++(-5, 0);
    \draw[dashed] (B) -- ++(5,0);
    \draw[dashed] (B) -- ++(2, -3);
    \draw[dashed] (C) -- ++(2, 3);
    \draw[dashed] (C) -- ++(-2, 3);


    % Ajouter des flèches perpendiculaires avec des points d'interrogation
    %\draw[->, thick] (2, -0.1) -- (2, -0.7) node[anchor=north] {?};
    %\draw[->, thick] (3.8, 2) -- (4.3, 2.5) node[anchor=west] {?};
    \draw[<->, thick] (0.3, 5) -- (0.3 +0.5, 5 +0.5) node[anchor=west] {?};
    %\draw[->, thick] (0.8, 1.5) -- (0.3, 2) node[anchor=east] {?};

    % Tracer les surfaces colorées (facultatif pour ce schéma)
    \fill[blue!20, opacity=0.7] (A) -- (P) -- (C) -- cycle;
    \fill[green!20, opacity=0.7] (P) -- (B) -- (C) -- cycle;
    \fill[pattern=north west lines, pattern color=red, opacity=0.2] (A) -- (B) -- (P) -- cycle;

    % Marquer les points
    \filldraw[red] (A) circle (2pt) node[anchor=north] {A};
    \filldraw[red] (B) circle (2pt) node[anchor=north] {B};
    \filldraw[red] (C) circle (2pt) node[anchor=south] {C};
    \filldraw[blue] (P) circle (2pt) node[anchor=west] {P};

    % Ajouter des annotations pour la surface négative (facultatif)
    %\node at (2.5, 1.2) [anchor=east] {
     %   \begin{tikzpicture}[baseline]
    %        \fill[pattern=north west lines, pattern color=red, opacity=0.7] (0,0) rectangle (0.5,0.2);
    %    \end{tikzpicture}
    %    \hspace{0.1cm} Surface négative
    %};
    %\draw[->, thick] (2.5, 1.2) -- (2.5, 0.5);

\end{tikzpicture}

\begin{itemize}
    \item La surface bleu est à l’extérieur, ce qui est vérifié par le fait que le point P ne se situe pas du même coté de la droite (formée par) AC que le point B.
    \item La surface verte est à l’extérieur, ce qui est vérifié par le fait que le point P ne se situe pas du même coté de la droite BC que le point A.
    \item La surface hachurée en rouge est à l'intérieur, ce qui est vérifié par le fait que le point P se situe du même coté de la droite AB que le point C.
\end{itemize}


\end{document}
