\chapter*{Conclusion}
\addcontentsline{toc}{chapter}{Conclusion}


Ce stage au CERFACS m'a vraiment plu. Il m'a permis de développer et renforcer beaucoup de connaissances dans le milieu de l'informatique, mais aussi sur le plan humain. J'ai pu découvrir comment se déroulais la vie en laboratoire de recherche.

L'un des principaux résultats de ce stage a été l'implémentation globalement réussie de la méthode d'interpolation linéaire, qui s'est avérée généralement précise que la méthode de Pondération Inverse à la Distance (IDW) précédemment codée.
Coder l'interpolation linéaire jusqu'en 3D n'a pas été très difficile ou chronophage, mais bien l'intégrer au code déjà existant et faire les tests l'étais plus. Heureusement que j'avais le support de mon maître de stage pour tout cela.
Ce travail aussi amélioré les performances du code.
Il pourrait aussi aider à l'intégration de méthodes d'ordre supérieur par l'état de l'art présent dans ce rapport et certaines fonctions réutilisables.
J'ai ressenti tout au long de mon stage l'importance de faire un travail qui puisse être continué plus tard par un autre chercheur.

Au-delà des aspects techniques, ce stage m'a offert un bel apercu du travail de chercheur.

Enfin, ce stage a été une expérience enrichissante qui a conforté mon intérêt pour les méthodes numériques appliquées et la recherche en calcul scientifique. Je suis reconnaissant pour l'accompagnement que j'ai reçu tout au long de cette expérience et pour les opportunités de développement personnel et professionnel qu'elle m'a offert.
