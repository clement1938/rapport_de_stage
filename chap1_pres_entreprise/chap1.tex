\chapter{Le CERFACS}
%\chapter{CERFACS : Le fondement des logiciels de simulation numérique}
%\chapter*{Le CERFACS : Le fondement des logiciels de simulation numérique}
%\addcontentsline{toc}{chapter}{Le CERFACS : Le fondement des logiciels de
                                %simulation numérique}

Le Centre Européen de Recherche et de Formation Avancée en Calcul Scientifique est un laboratoire de recherche privé avec pour actionnaires Airbus, le CNES (Centre d'Études Spatiale), EDF, Météo France, l'ONERA (Office National d'Études et de Recherche Aérospatiales), Safran et TotalEnergies. Il a pour but de développer la simulation numérique par le calcul haute performance (HPC) pour ses actionnaires, mais aussi de faire de la recherche et de former des ingénieurs, chercheurs et doctorants. Il a été créé en 1988 sous le statut de GIP (Groupement d’intérêt Public), pour devenir une société civile en 1996 et depuis 2021, le CERFACS est une SAS (Société par Actions Simplifiées).

Les deux bâtiments sont situés au Météopole, dans la partie Ouest de Toulouse. Environ 170 personnes y travaillent, dont 20 \% de femmes, 50\% de doctorants et 20\% d'étrangers dont la moitié ne sont pas Européens.

Physiciens, mathématiciens, informaticiens, numériciens et Data Scientistes y travaillent dans quatre équipes :

% A mettre à jour

- Algo-Coop (Algorithmes Parallèles \& sCientifics sOftware Operational Performances)

- CSG (Équipe Informatique et Support Utilisateur)

- CFD (Mécanique des fluides numérique) % (Computation Fluid Dynamics en anglais)

- GLOBC (Modélisation du climat et de son changement global)


%\begin{comment}
La CAO permet de faire les plans numériques d'un avion par exemple, afin de s'assurer que toutes les pièces fabriquées vont bien s'imbriquer entre elles. Mais cela permet aussi de faire des simulations numériques pour prévoir à l'avance la tenue structurelle, les forces aérodynamiques, le volume sonore, etc. sans faire d'onéreux tests.
%Simuler les écoulements autour des avions, dans ses moteurs, dans les bâtiments ou encore à l'échelle du globe %terrestre est un défis ...
L'équipe CFD (Computational Fluid Dynamics) est la plus grande du CERFACS. Elle se focalise sur la simulation des écoulements et de la combustion en développant des méthodes numériques avancées et en les appliquant aux avions, fusées, hélicoptères, moteurs, turbomachines, etc. Les liens de l’équipe CFD avec les autres équipes du CERFACS comme GLOBC ou PAE  sont forts, car ces équipes utilisent aussi la CFD de façon intensive pour prévoir le changement climatique ou l’effet de l’aviation sur l’environnement : en effet, derrière ces thèmes, on retrouve en premier les équations qui régissent les écoulements des fluides.
%\end{comment}

Le CERFACS héberge des supercalculateurs (Kraken et Calypso) et un serveur sur lequel se trouve l'intranet avec toutes les ressources nécessaires aux employés.
Nous y trouvons notamment un lien vers une page QVT\footnote{QVT : Qualité de Vie au Travail}\dots
Et un lien vers la page du CSE\footnote{CSE : Comité Social et Économique}\dots avec énormément de ressources.
Le CSSCT\footnote{CSSCT : Commissions santé, sécurité et conditions de travail} et le DUERP\footnote{DUERP : Document Unique d’Évaluation des Risques Professionnels} devrait obligatoirement y être présent selon le site du gouvernement.
Je n'ai par réussi à trouver le document relatif à la RSE\footnote{RSE : Responsabilité Sociétale des Entreprises}.
Il existe un document "Plan du management de la qualité"


La question de l’environnement est très présente au CERFACS. Les employés sont poussés à venir à vélo. Ils donnent la prime (obligatoire pour les entreprises) pour chaque kilomètre fait à vélo sur le trajet habitation-travail-et vis-versa. Il y a un atelier de réparation et des emplacements sécurisés pour les vélos.

... Ils ont par exemple lancé un appel au volontariat pour composer le Comité de Pilotage pour la Prévention des Violences Sexistes et Sexuelles au Travail suite à une réunion d'information réalisée en collaboration avec la médecine du travail.

Le cadre de travail m'a relativement plus, être assis dans un bureau climatisé avec une vue sur la campagne n'est pas désagréable. Un des seuls risques à ce type de travail est une mauvaise position qui peut entraîner des problèmes de dos, épaules, ... Pour palier cela il y a des affiches dans chaque bureau indiquant la position à avoir et des exercices. Ces mêmes informations apparaissent dans le 'livret d'accueil stagiaire' qui m'a été donné le premier jour.
Il y a une cantine sur le site à quelques minutes à pied, où la plupart des chercheurs et thésards vont manger les midis.



\begin{comment}
• L’entreprise s’est-elle engagée dans une démarche RSE ?
• Si c’est le cas, comment en évaluez-vous l’impact dans les domaines suivants ?
– Emploi et relations employeur-employé
– Conditions de travail et protection sociale
– Dialogue social
– Santé et sécurité au travail
– Développement des ressources humaines
• Quels objectifs le volet environnemental de sa démarche RSE favorise-t-il ?
– La réduction des impacts environnementaux
– L’économie des ressources
– Le traitement des déchets
– La gestion des déplacements
– Autre…
• Quel était l’impact de la démarche RSE et de ses objectifs sur votre travail en tant que stagiaire ?
• La parité homme-femme (rémunérations, carrières, embauches, formation) est-elle une réalité dans l’entreprise ? Sinon, des mesures ont-elles été prises pour l’atteindre ? Lesquelles ?
• L’entreprise a-t-elle pris des engagements éthiques ? Communique-t-elle une charte de bonne conduite ?
• Quels sont les risques sociaux et environnementaux que pourraient engendrer les activités de l’entreprise où vous avez effectué votre stage ? Quelles mesures celle-ci a-t-elle adoptées pour éviter ou amoindrir ces risques ?
• Qui sont les acteurs principaux dans les domaines de la prévention des risques et de la protection de la santé au travail (fonctions, missions, etc.) ?
• Existe-t-il une CSSCT ? Quelle est sa composition ? Quel est son rôle ?
• Existe-t-il un DUERP ? Sous quelle forme se présente-t-il ? Est-il régulièrement mis à jour et par qui ? Existe-t-il un suivi de la réalisation du programme d’actions ?
• Dans l’accomplissement des tâches qui vous ont été confiées, avez-vous observé des situations potentiellement dangereuses ?
Annexe IV. Développement durable, RSE, CSE
Charte des écrits et des oraux de l’IPSA – 2023-2024 – 41 / 46
• Avez-vous reçu une formation particulière relative à votre poste de travail ? Qui vous a dispensé cette formation (poste, fonction, etc.) ?
• Y a-t-il, dans l’entreprise, des équipements de protection collective ? Lesquels ?
• Y a-t-il des équipements de protection individuelle (lunettes, blouse, casque…) ? Les avez-vous
portés ? À quelles occasions ? Lesquels ?
• Vous a-t-on relaté des incidents liés à votre poste de travail ou à d’autres postes dans l’entreprise ?
• Si c’est le cas, des mesures ont-elles été prises à la suite de ces incidents ? Lesquelles ?
\end{comment}