\chapter{Présentation de l'entreprise}
\chapter*{Le CERFACS : Le fondement des logiciels de simulation numérique}
%\addcontentsline{toc}{chapter}{Le CERFACS : Le fondement des logiciels de
                                %simulation numérique}

Le Centre Européen de Recherche et de Formation Avancée en Calcul Scientifique
est un laboratoire de recherche privé avec pour actionnaires Airbus,
le CNES (Centre d'Études Spatiale), Météo France,
l'ONERA (Office National d'Études et de Recherche Aérospatiales), Safran
et TotalEnergie.

Il est constitué de quatre équipes :

- Algo-Coop (Algorithmes Parallèles \& sCientifics sOftware Operational Performances)

- CSG (Equipe Informatique et Support Utilisateur)

- CFD (Mécanique des fluides numérique)

- GLOBC (Modélisation du climat et de son changement global)

\begin{comment}
Simuler les écoulements autour des avions, dans les moteurs, dans les bâtiments…
La CFD (Computational Fluid Dynamics) est la plus grosse équipe  du CERFACS. Elle se focalise sur la simulation des écoulements en développant des méthodes numériques avancées et en les appliquant aux avions, fusées, hélicoptères, moteurs de voitures, turbines, etc. Les liens de l’équipe CFD avec les autres équipes du CERFACS comme GLOBC ou PAE  sont forts car ces équipes utilisent aussi la CFD de façon intensive pour prévoir le changement climatique ou l’effet de l’aviaton sur l’environnement: en effet, derrière ces thèmes, on retrouve en premier les équations qui régissent les écoulements des fluides.

Calculez ce moteur avant de le construire !
L’équipe CFD développe des outils essentiels dans de nombreux domaines applicatifs avec un leitmotiv bien connu aujourd’hui dans l’industrie: calculons les systèmes (avions, moteurs…) avant de les construire. Ceci permet de diminuer le coût des essais d’un ordre de grandeur dans de nombreux cas et l’équipe CFD du CERFACS est au centre de ces travaux avec ses actionnaires. La CFD est aussi le domaine où les calculs à haute performance sont le plus présents: la CFD se fait avec des maillages contenant des milliards de points sur des machines contenant des millions de coeurs. Réaliser ces calculs de façon fiable et efficace est un autre thème central sur lequel le CERFACS est reconnu comme un acteur majeur.
\end{comment}



\begin{comment}
• L’entreprise s’est-elle engagée dans une démarche RSE ?
• Si c’est le cas, comment en évaluez-vous l’impact dans les domaines suivants ?
– Emploi et relations employeur-employé
– Conditions de travail et protection sociale
– Dialogue social
– Santé et sécurité au travail
– Développement des ressources humaines
• Quels objectifs le volet environnemental de sa démarche RSE favorise-t-il ?
– La réduction des impacts environnementaux
– L’économie des ressources
– Le traitement des déchets
– La gestion des déplacements
– Autre…
• Quel était l’impact de la démarche RSE et de ses objectifs sur votre travail en tant que stagiaire ?
• La parité homme-femme (rémunérations, carrières, embauches, formation) est-elle une réalité dans l’entreprise ? sinon, des mesures ont-elles été prises pour l’atteindre ? lesquelles ?
• L’entreprise a-t-elle pris des engagements éthiques ? communique-t-elle une charte de bonne conduite ?
• Quels sont les risques sociaux et environnementaux que pourraient engendrer les activités de l’entreprise où vous avez effectué votre stage ? quelles mesures celle-ci a-t-elle adoptées pour éviter ou amoindrir ces risques ?
• Qui sont les acteurs principaux dans les domaines de la prévention des risques et de la protection de la santé au travail (fonctions, missions, etc.) ?
• Existe-t-il une CSSCT ? quelle est sa composition ? quel est son rôle ?
• Existe-t-il un DUERP ? sous quelle forme se présente-t-il ? est-il régulièrement mis à jour et par qui ? existe-t-il un suivi de la réalisation du programme d’actions ?
• Dans l’accomplissement des tâches qui vous ont été confiées, avez-vous observé des situations potentiellement dangereuses ?
Annexe IV. Développement durable, RSE, CSE
Charte des écrits et des oraux de l’IPSA – 2023-2024 – 41 / 46
• Avez-vous reçu une formation particulière relative à votre poste de travail ? qui vous a dispensé cette formation (poste, fonction, etc.) ?
• Y a-t-il, dans l’entreprise, des équipements de protection collective ? lesquels ?
• Y a-t-il des équipements de protection individuelle (lunettes, blouse, casque…) ? les avez-vous
portés ? à quelles occasions ? lesquels ?
• Vous a-t-on relaté des incidents liés à votre poste de travail ou à d’autres postes dans l’entreprise ?
• Si c’est le cas, des mesures ont-elles été prises à la suite de ces incidents ? lesquelles ?
\end{comment}