
\cleardoublepage

\paragraph{Résumé}

\vspace{0,5cm}

Ce rapport de stage présente le travail effectué au CERFACS, un institut de recherche axé sur le calcul haute performance (HPC).

Destiné à l'IPSA, ainsi qu'aux utilisateurs et développeurs du code d'interpolation de la librairie Antares utilisée pour le post-traitement de simulations numériques, ce document propose une vue d'ensemble des activités menées.

Il débute par une présentation générale du CERFACS, avant de détailler le stage.
Ce dernier inclut une introduction à la librairie Antares et à son utilisation pour l'interpolation, suivie d'un état de l'art des différentes méthodes d'interpolation susceptibles d'être intégrées dans la librairie.
Une description de l'implémentation de la méthode linéaire est ensuite fournie, avec un focus sur les optimisations réalisées.
Enfin, les résultats de la comparaison entre la méthode de Pondération Inverse à la Distance (IDW) précédemment codée et la méthode linéaire implémentée sont exposés.

\paragraph{Summary}

\vspace{0,5cm}

This internship report presents the work carried out at CERFACS, a research institute focused on High-Performance Computing (HPC).

Intended for, as well as the users and developers of the interpolation code in the Antares library used for post-processing numerical simulations, this document provides an overview of the activities undertaken.

It begins with a general presentation of CERFACS, followed by a detailed account of the internship.
The latter includes an introduction to the Antares library and its use for interpolation, followed by a state-of-the-art review of various interpolation methods that could potentially be integrated into the library.
A description of the implementation of the linear method is then provided, with a focus on the optimizations performed.
Finally, the results of the comparison between the previously coded Inverse Distance Weighting (IDW) method and the implemented linear method are presented.